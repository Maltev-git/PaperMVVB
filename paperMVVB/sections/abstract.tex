
\title{Solar wind predictions for the Parker Solar Probe orbit}
\subtitle{Near-Sun extrapolations derived from an empirical solar wind model based on Helios and OMNI observations}

\author{M.~S.~Venzmer
\and V.~Bothmer}

\institute{University of Goettingen, Institute for Astrophysics, Friedrich-Hund-Platz~1, 37077~Göttingen, Germany}

\date{First draft 9 August 2016; first changes 21 June 2017; second changes 17 July 2017; third changes 17 August 2017; received date; accepted date}

\abstract
{In view of the planned near-Sun spacecraft mission Parker~Solar~Probe (PSP) (formerly Solar~Probe~Plus) the solar wind environment for its prime mission duration (2018--2025) and down to its intended closest perihelion (\num{9.86}~solar radii) is extrapolated using in situ data. The PSP mission will be humanity's first in situ exploration of the solar corona. Visiting this yet uncharted region is of special interest, because it will help answer hitherto unresolved questions on the heating of the solar corona and the source and acceleration of the solar wind and solar energetic particles. The solar wind extrapolation of this study is performed within the project Coronagraphic German And US Solar~Probe Survey (CGAUSS) which is the German contribution to the PSP mission as part of the Wide field Imager for Solar PRobe (WISPR).}	%context
{We present an empirical solar wind model for the inner heliosphere which is derived from Helios and OMNI in situ data. The German-US space probes Helios~1 and Helios~2 flew in the 1970s and observed solar wind in the ecliptic within heliocentric distances of \SIrange{0.29}{0.98}{\au}. The OMNI database consists of multi-spacecraft intercalibrated in situ data obtained near \SI{1}{\au}. The solar wind model is used together with sunspot number predictions to estimate the frequency distributions of major solar wind parameters PSP will encounter during its mission.}	%aims
{The model covers the solar wind's magnetic field strength and its plasma parameters proton velocity, density and temperature.
Their individual frequency distributions are represented with lognormal functions. In addition, we also consider the velocity distribution's bi-componental shape, consisting of a slower and a faster part. The model accounts for solar activity and for solar distance dependency by shifting of these lognormal distributions. We compile functional relations to solar activity by correlating and fitting the frequency distributions with the sunspot number (SSN), using almost five solar cycles of OMNI data. Further, based on the combined data set from both Helios probes, the parameters' frequency distributions are fitted with respect to solar distance to obtain power law dependencies. Finally, by combining the found solar cycle and solar distance relations, we obtain a simple dynamical solar wind model for the inner heliosphere, confined to the ecliptic region.}	%methods
{The inclusion of SSN predictions and the extrapolation to the PSP perihelion region enables us to estimate the solar wind environment for PSP's planned trajectory during its mission duration. The estimated solar wind median values during PSP's first perihelion are \SI{87}{\nT}, \SI{340}{\km\per\s}, \SI{4015}{\per\cm\cubed} and \SI{503000}{\K}. The modeled values for PSP’s closest perihelia are \SI{943}{\nT}, \SI{290}{\km\per\s}, \SI{9733}{\per\cm\cubed} and \SI{1930000}{\K}, where these velocity and temperature values are clearly overestimated in comparison with existing observations.}	%results
{This empirical model shows that solar wind acceleration and heating processes below about 20~solar radii limit a simple back-extrapolation from existing in situ measurements.}	%conclusions

\keywords{solar wind -- sun: heliosphere -- sun: corona}

\maketitle

%\titlerunning{Solar wind extrapolation to PSP orbit}
%\authorrunning{Venzmer \& Bothmer}



%\SI{0.0459}{\au}
%extrapolated ranges \SI{95.45}{\percent} of the solar wind parameter values lie within. The ranges are for the magnetic field strength \SIrange{522}{2725}{nT}, for the velocity \SIrange{209.4}{506}{\km\per\s}, for the density \SIrange[scientific-notation=fixed, fixed-exponent=3, range-units=brackets]{1.319e3}{1.72e4}{\per\cm\cubed} and for the temperature \SIrange[scientific-notation=fixed, fixed-exponent=6, range-units=brackets]{4.18e5}{1.59e7}{\K}.
%These ranges are consistent with... and differ with... 