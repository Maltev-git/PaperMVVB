
\title{Solar wind prediction for the Solar Probe Plus orbit}
\subtitle{Near-Sun extrapolations derived from an empirical solar wind model based on Helios and OMNI observations}

\author{M.~S.~Venzmer
\and V.~Bothmer}

\institute{University of Goettingen, Institute for Astrophysics, Friedrich-Hund-Platz~1, 37077~Göttingen, Germany}

\date{First draft 9 August 2016; received date; accepted date }

\abstract
{In view of the planned near-Sun spacecraft mission Solar~Probe~Plus (SPP) the solar wind environment for its mission duration (2018--2025) and down to its intended closest perihelion (\num{9.86}~solar radii) is extrapolated using in situ data. This yet uncharted region is of special interest because the critical Alfvénic and sonic surfaces are believed to lie within and thus still part of the solar wind acceleration as well.}	%context
{We present an empirical solar wind model for the inner heliosphere which is derived from Helios and OMNI in situ data. The space probes Helios~1 and Helios~2 flew in the 1970s and observed solar wind in the ecliptic within heliocentric distances of \SIrange{0.29}{0.98}{\au}. The OMNI data set consists of multi-spacecraft intercalibrated in situ data from \SI{1}{\au}. This model is used together with sunspot number predictions to estimate the frequency distributions of major solar wind parameters SPP will encounter on its mission.}	%aims
{The model covers the solar wind's magnetic field strength and its plasma parameters proton density, velocity and temperature.
Their individual frequency distributions are represented with lognormal functions. In addition, we also consider the velocity distribution's bi-componental shape, consisting of a slower and a faster part. The model accounts for solar activity and for solar distance dependency by shifting of these lognormal distributions. We compile functional relations to the solar activity by correlating and fitting the frequency distributions with the sunspot number (SSN), using almost five solar cycles of OMNI data. Further, based on the combined data set from both Helios probes, the parameters' frequency distributions are fitted over solar distance to obtain exponential dependencies. Combining both the solar cycle and the solar distance relations, we get a simple dynamical solar wind model for the inner heliosphere, confined to the ecliptic.}	%methods
{The inclusion of SSN predictions and the extrapolation to the SPP perihelion region enables us to estimate the solar wind environment at SPP's planned orbital positions during its mission time.}	%results
{}	%conclusions
%\SI{0.0459}{\au}

\keywords{solar wind -- sun: heliosphere -- sun: corona}

\maketitle

%\titlerunning{Solar wind extrapolation to SPP orbit}
%\authorrunning{Venzmer \& Bothmer}




%extrapolated ranges \SI{95.45}{\percent} of the solar wind parameter values lie within. The ranges are for the magnetic field strength \SIrange{522}{2725}{nT}, for the velocity \SIrange{209.4}{506}{\km\per\s}, for the density \SIrange[scientific-notation=fixed, fixed-exponent=3, range-units=brackets]{1.319e3}{1.72e4}{\per\cm\cubed} and for the temperature \SIrange[scientific-notation=fixed, fixed-exponent=6, range-units=brackets]{4.18e5}{1.59e7}{\K}.
%These ranges are consistent with... and differ with... 